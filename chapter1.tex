\chapter{INTRODUCTION}

{\baselineskip=2\baselineskip

\section{Background of the Study}

Livestock weight estimation is a fundamental aspect of animal husbandry, directly influencing decisions regarding feeding schedules, growth monitoring, health management, and market readiness \citep{wang2024review}. Among various livestock species, pigs are a primary focus due to their significant contribution to global meat production. Pigs account for approximately 36 percent of the world's meat consumption, surpassing other livestock such as cattle and poultry, making them an economically critical species \citep{FAO2022}. Accurate weight estimation in pigs ensures optimal feed conversion rates, reduced production costs, and improved animal welfare, which are essential for sustainable farming practices \citep{terence2024systematic}.

In traditional farming methods, weighing scales are the standard tool for livestock weight measurement. However, these methods are often labor-intensive, time-consuming, and stressful for animals, potentially leading to injuries and a decline in productivity \citep{faucitano2018transport}. For smaller farms, the high costs and maintenance requirements of weighing systems make them less feasible. To address these challenges, farmers and researchers have explored alternative methods, such as using a measuring tape or string to estimate pig weight based on body measurements like heart girth and length \citep{ThePigSite}. While these methods are cost-effective and less stressful, they are prone to inaccuracies due to human error and inconsistent measurement techniques.

Advancements in technology, particularly computer vision (CV) and machine learning (ML), have introduced more efficient and precise methods for weight estimation. Depth-sensing devices like Microsoft Kinect offer a non-invasive and cost-effective solution by capturing 3D data to estimate body weight with high accuracy \citep{pezzuolo2018barn}. Compared to manual measurement methods or even traditional scales, these technologies significantly reduce labor, improve accuracy, and minimize stress on animals. Moreover, the integration of 3D imaging and ML algorithms enables the creation of automated systems that adapt to diverse farm environments, making them ideal for small and medium-sized farms \citep{gjergji2020deep}.

Manual methods such as using measuring tapes involve wrapping a tape or string around the pig's heart girth and measuring its body length, followed by applying a formula to estimate weight \citep{ThePigSite}. While accessible and inexpensive, this method is laborious and lacks precision, especially when dealing with active or uncooperative animals. Traditional weighing scales, though accurate, require direct handling of pigs, posing risks to animal welfare and farmworkers \citep{dickinson2013automated}. In contrast, advanced systems leveraging depth-sensing technology provide non-invasive and automated solutions. For instance, Kinect sensors capture detailed 3D images, enabling precise calculations of body volume and weight without requiring physical contact with the animal \citep{pezzuolo2018barn}. These systems mitigate stress, enhance accuracy, and save labor, making them a promising alternative to traditional methods. However, the initial cost and need for technical expertise can pose barriers to adoption, especially for resource-constrained farms.

Accurately estimating pig weight is critical for achieving optimal feed-to-weight conversion ratios, a key determinant of farm profitability. Weight monitoring also supports health assessments, helping detect early signs of disease or malnutrition, which can otherwise lead to significant economic losses. Furthermore, precise weight data allows farmers to determine the ideal market timing, maximizing revenue and ensuring meat quality standards are met \citep{terence2024systematic}.

This study aims to develop a system for pig weight estimation by leveraging Kinect V1's depth-sensing capabilities. The proposed system seeks to reduce labor, improve measurement precision, and enhance animal welfare. This research contributes to the broader adoption of sustainable, efficient farming practices by addressing the practical challenges of traditional and manual weight estimation methods. The findings may pave the way for integrating advanced depth-sensing technologies into modern livestock management systems, supporting the transition toward precision agriculture and smart farming practices.

 

%-----------------------------------------------------------------------------------------------

%-----------------------------------------------------------------------------------------------------------------------------------
\section{Statement of the Problem}

Accurate weight estimation is a cornerstone of efficient pig farming, influencing critical decisions such as feed management, health monitoring, growth tracking, and market readiness. Traditional weighing methods, which involve manually moving pigs to weighing scales, are not only labor-intensive and time-consuming but also stressful for the animals. This stress can adversely affect the pigs’ well-being, reducing their productivity and potentially impacting meat quality \citep{li2014estimation}. The labor-intensive nature of manual weighing also increases operational costs, posing a challenge for farmers, particularly in small- to medium-scale farms where resources are limited.

To circumvent these challenges, farmers often resort to alternative methods, such as using a measuring tape or string to estimate pig weight. This method involves measuring dimensions like heart girth and body length, then applying a weight estimation formula \citep{ThePigSite}. While this approach is cost-effective and non-invasive, it is prone to significant inaccuracies due to human error, variability in measurement techniques, and difficulty in handling active or uncooperative animals. These inaccuracies can lead to suboptimal feeding strategies, delayed health interventions, and missed opportunities for maximizing market profitability.

In recent years, advancements in imaging technologies have paved the way for automated, non-invasive weight estimation methods. Single-camera systems and monocular vision techniques, while offering some improvements, often fail to capture precise depth information, which is crucial for accurate volume and weight estimation \citep{pezzuolo2018barn}; \citep{kollis2007weight}. These systems are further limited by environmental factors such as lighting variations and animal movement, which compromise their reliability and usability in real-world farm settings.

Multi-camera setups have emerged as a more accurate alternative, enabling the capture of detailed 3D data necessary for precise weight calculations \citep{dohmen2022computer}. However, these setups are often prohibitively expensive and complex, requiring specialized equipment, significant computational resources, and skilled personnel to operate. This makes them inaccessible to many farmers, especially those in resource-constrained environments. Additionally, while machine learning algorithms and other modeling techniques have been applied to refine weight estimation processes, their effectiveness is limited by the quality of input data, which often suffers from incomplete 3D reconstructions or inconsistencies in 2D images.

A promising yet underexplored solution lies in utilizing Microsoft Kinect V1, a cost-effective and readily available depth-sensing camera originally developed for gaming applications. The Kinect V1 captures depth data by projecting an infrared (IR) pattern onto the surface of an object and analyzing the distortions to calculate depth values. This enables the generation of detailed 3D point clouds that can be used to accurately measure body volume and, subsequently, estimate weight \citep{Zhang2012}. Despite its potential, the Kinect V1 remains underutilized in the domain of livestock management, and existing research has not fully explored its capabilities for precise and non-invasive weight estimation.

This gap in the literature highlights the need for a practical and accessible system that leverages Kinect’s depth-sensing technology for weight estimation in pigs. Unlike traditional methods or multi-camera systems, the Kinect V1 offers a simpler, more affordable solution capable of capturing high-quality depth data in real-time. By automating the weight estimation process, the Kinect V1 can significantly reduce labor requirements, minimize stress on animals, and improve the accuracy of measurements.

This research aims to bridge this gap by leveraging the depth-sensing capabilities of the Kinect V1 to develop a cost-effective and automated system for pig weight estimation. Unlike previous approaches that rely on more complex and expensive setups, the Kinect-based system provides detailed depth information for accurate body volume measurements. By addressing the challenges of traditional and existing imaging methods, this study seeks to create a solution that minimizes labor, reduces stress on animals, and enhances measurement precision. The findings from this research could revolutionize weight monitoring practices in pig farming, particularly for small and medium-scale farms, and contribute to the advancement of precision agriculture and smart farming techniques.


\section{Objectives of the Study}

The main objective of this study is to develop a non-invasive Pig Weight Estimation System that utilizes depth imaging, specifically:
\begin{enumerate}
	\item To utilize CNN and LightGBM for model training and validation to analyze depth data for pig weight estimation. 
	\item To develop a web application for the Pig Weight Estimation System.
	\item To assess the accuracy of the Kinect-based pig weight estimation system.
\end{enumerate}

\section{Significance of the Study}

The results of this study would benefit the following: 

\textit{Academe.} This study will contribute to the existing knowledge on the application of computer vision technology in the industry, particularly in livestock management. This study could serve as a valuable reference for academic institutions for agriculture, computer science, engineering, and computer vision technology. The findings of this study can also be an inspiration for further research in these fields of study.

\textit{Business Owners.} This study can offer business owners, particularly those in the agricultural and livestock sectors a more efficient and data-driven method of managing their livestock operations. The results of this study can help automate tasks such as livestock monitoring, feed rationing, and health checks. This can reduce labor costs and improve decision-making processes. Additionally, this innovation can encourage entrepreneurship in tech-driven agriculture, creating business opportunities.

\textit{Livestock Caretakers.} The results of this study can aid in the adoption of computer vision technology in livestock agriculture and can assist livestock caretakers in optimizing feed rations, monitoring growth rates, and improving overall herd management. Thus can ensure better animal health and welfare, reduce workload, and enhance productivity by providing data-driven insights into the daily operations of the livestock.

\textit{Government Organizations.} Government organizations can utilize the study’s findings to develop strategies that promote the modernization of agriculture through technological adoption. By using computer vision technologies, government organizations can improve farm-to-market systems and boost overall agricultural productivity.

\textit{Veterinarians.} Results from this study could assist veterinarians involved in livestock by creating tools that enable veterinarians to remotely monitor, reducing the need for frequent on-site visits. It can also help in establishing preventive care strategies by being able to identify early signs of illness through variations in weight, improving animal welfare.

\textit{Agriculture Technology Developers.} Companies and developers in the field of agricultural technology can use the findings of this study to design and enhance their products. The research can inform them about the specific needs of livestock management and provide insights into how computer vision solutions can be tailored to address these requirements, resulting in more market-relevant and effective innovations.

\textit{Future Researchers.} This study can be used as a future reference for researchers who plan to engage in the same field of study. It can potentially be adapted for other agricultural applications, leading to increased efficiency and productivity in the sector. The methodologies found in the study can guide subsequent research, encouraging innovation and advancement in agriculture technology.

\section{Scope and Limitations}

This research focuses on the non-invasive estimation of weight in Landrace pigs, a breed commonly utilized in commercial pig farming in the Philippines due to its favorable traits and adaptability to the tropical climate \citep{Manez2020}. Depth data for this study was collected using a Microsoft Kinect V1 camera sensor positioned at a fixed height of 1.9 meters from the floor, capturing a top-view perspective of the pigs (primarily showing their backs). The data was gathered in small backyard-type pig farms located in Cagayan de Oro City, Philippines. While data was collected with consistent lighting, this study will analyze depth data of Landrace pigs in various postures using Convolutional Neural Networks (CNNs). The Landrace pigs in this study typically range from 8 to 22 kg in weight, corresponding to an age of approximately 1 to 2 months and exhibiting an average daily gain (ADG) of 0.7–1.0 kg/day under optimal feeding conditions.

This study will utilize CNNs to analyze depth images, captured from a top-down view at a consistent height of 1.9 meters, and extract precomputed features, including:
\begin{itemize}
	\item \textbf{Pixel Size}: Total pixels in the segmented region ($shape[0] \times shape[1]$).
	\item \textbf{Non-Zero Pixel Count}: Pixels with $depth > 0$ ($np.count\_nonzero$).
	\item \textbf{Pixel-to-Non-Pixel Ratio}: Ratio of non-zero to total pixels.
	\item \textbf{Standard Deviation of Depth}: Depth variability ($np.std$).
	\item \textbf{Mean Depth}: Average depth of non-zero pixels ($np.mean$ where $depth > 0$).
	\item \textbf{Pixel Ratio}: Non-zero pixel count relative to 1280$\times$480.
	\item \textbf{Volume Proxy}: Sum of depth values ($np.sum$), approximating 3D volume.
	\item \textbf{Aspect Ratio}: Bounding box width-to-height ratio ($shape[1] / shape[0]$).
	\item \textbf{Perimeter}: Sum of contour arc lengths ($cv.arcLength$).
\end{itemize}
The primary goal is to train a CNN model capable of estimating pig weight based on these 3D body volume measurements derived from the depth images captured by the Microsoft Kinect V1 sensor from a fixed overhead position. For the purpose of evaluating the system's real-time accuracy in a controlled environment, a local server implementation will be used. This approach allows for focused testing and immediate feedback on the model's performance without the complexities of a wider network deployment or internet dependency, which can introduce additional variables and potential points of failure during this initial validation phase. The current study's weight estimation will be limited to the 8-22 kg range due to the data collected under these specific conditions. However, the developed model will serve as a foundational step for future research aimed at scaling the system to accommodate a wider range of pig weights and potentially exploring different camera angles or deployment in more diverse farm environments.




\section{Definition of Terms}

\begin{description}

	\item[Artificial Intelligence] -
	A set of technologies that enable computers to perform a variety of advanced functions, including the ability to see, understand, and translate spoken and written language, analyze data, make recommendations, and more.
	
	\item[Computer Vision] 
	A field of computer science that focuses on enabling computers to identify and understand objects and people in images and videos.
	
	\item[Deep Learning] 
	A subset of machine learning that uses multilayered neural networks, called deep neural networks, to simulate the complex decision-making power of the human brain.
	
	\item[Depth Imaging] 
	A technique that captures the distance between a sensor (like a depth camera) and objects in its view, creating a 3D representation of the scene. Depth imaging provides additional data on the spatial positioning of objects.
	
	\item[Landrace] 
	A domesticated, locally adapted, traditional variety grown by farmers and their successors since ancient times.
	
	\item[Image Processing] 
	Techniques used to enhance or extract information from images, are crucial for analyzing data from the Kinect sensor.
	
	\item[Machine Learning] 
	A subset of artificial intelligence that involves the use of algorithms to learn from data and make predictions or decisions based on that data.
	
	\item[Microsoft Kinect Sensor] 
	A motion-sensing input device that uses a camera to capture depth data and track movements.
	
	\item[Pig Weight Estimation] 
	The process of determining the weight of pigs uses various methods, in this case, image processing and machine learning.

\end{description}

}
