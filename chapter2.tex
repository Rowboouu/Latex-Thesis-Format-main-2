\chapter{REVIEW OF RELATED LITERATURE}
{\baselineskip=2\baselineskip
This literature review explores current weight estimation techniques in livestock management, highlighting the advancements and limitations of traditional methods while introducing emerging technologies, particularly depth-sensing devices like the Kinect sensor. Understanding the shift from conventional approaches to tech-driven solutions provides insight into how these innovations can improve accuracy, efficiency, and scalability in livestock farming.

%-----------------------------------------------------------------------------------------------------------------------
\section{Weight Estimation Techniques}
In animal production, livestock body weight is a significant and widely used feature that has an impact on feed consumption, breeding potential, social behavior, energy balance, and overall farm management \citep{wang2024review}. It may be used indirectly in the assessment of health and welfare status \citep{dikmen2012effect}. There are two main approaches to measuring body weight in livestock: (1) direct approaches using scales, and (2) indirect approaches based on relationships between body part measurement and body weight.

Direct weighing methods rely on weighing technologies such as partial-weight or full-weight industrial scales capable of supporting small, medium, or large livestock. Some companies provide passive-weighing solutions that integrate sensor-rich scaling systems such as GrowSafe of Canada, the Bosch Precision Livestock Platform of Germany, and Rice Lake Weighing Systems of Australia which are capable to measure, log results, and transmitting information over wired or wireless networks \citep{wang2024review} . Other companies such as Arvet CIMA Control Pig and CIMA Control Cow Scaling Systems of Spain developed custom-made scales that provide dynamic-weighing systems where animals are weighed while in motion using walk-through or step-over weighers \citep{rousing2004stepping}. While these devices are very accurate, their acquisition, intended purpose and operation size, repeated calibration and maintenance costs associated with their placement in high-temperature variability, and corrosive environments are significant and beyond the affordability and sustainability limits of small and medium size farms and even commercial operators\citep{dikmen2012effect}. It has been studied that removing animals from paddocks and holding areas and leading them to weighing stations is costly, stressful, and potentially harmful activity for animals and handlers alike and also inadvertently leads to animal weight loss or even death \citep{faucitano2018transport}. Moreover, since the weighing process is very laborious, the frequency of measurements is not sufficiently high to permit the use of body weight as an indicator for other traits. However, since the affordability of direct weighing methods may impede small producers \citep{dickinson2013automated}, researchers have developed indirect weighing methods represented by regression models that relate morphometric measurements and image features to body weight in livestock. The direct acquisition of morphometric measurements can be accomplished with the aid of technologies with various degrees of complexity, from measuring tapes and types to specialized software or manual, semi-automatic, or automatic measurements extrapolated from images obtained with electro-optical devices such as mono-2D, stereo-2D, 3D, ultrasound, and infrared sensors \citep{wang2024review}.

\subsection{Role of Technology in Agriculture}

Technological advancements have played a pivotal role in transforming the agricultural sector. From mechanization to digital innovations, these technologies have increased efficiency, productivity, and sustainability in farming practices. In particular, precision agriculture, which uses technology to optimize field-level management of crop farming, has revolutionized the way food is produced \citep{witten1993practical}. Precision farming technologies include GPS, soil sensors, and drones, allowing farmers to monitor and manage their crops with unprecedented accuracy.
In livestock farming, technology has led to the development of automated feeding systems, health monitoring tools, and weight estimation systems that reduce labor and enhance animal welfare \citep{gomez2021systematic}. For example, sensor-based systems for monitoring livestock health provide farmers with real-time data, enabling proactive interventions to prevent diseases and improve overall herd management \citep{neethirajan2021digital}.

Furthermore, AI and machine learning have started to play an increasingly significant role in agriculture. These technologies enable predictive analytics for crop yields and disease outbreaks, and more recently, they are being used in 3D object detection systems for animal weight estimation, such as the Kinect-based system for pig weight estimation proposed in this study \citep{farooqui2024precision}. Integrating AI and machine learning into agricultural practices is expected to address many challenges related to food security, sustainability, and labor shortages \citep{ng2023machine}.

For example, one study explored the use of 3D images captured from a zenithal viewpoint to estimate lambs' live weight. The researchers applied image processing techniques to extract features such as upper body area and average depth, demonstrating the potential of 3D imaging for livestock weight estimation \citep{samperio2021lambs}. Although this study did not use Kinect, it highlights the value of 3D data, which Kinect is well-equipped to provide, for accurate livestock weight estimation.

Another significant study used the Microsoft Kinect V1 depth camera to measure pig body dimensions and estimate their weight. The researchers found a strong correlation between the Kinect-based measurements and actual weights, with coefficients of determination (R²) exceeding 0.90 \citep{pezzuolo2018barn}. Similarly, Lao et al. also employed a Kinect V1 depth camera to extract body measurements from pigs, developing a regression model for weight estimation, while Liu et al. used a binocular vision system to collect 3D data and tested various modeling approaches, including linear, nonlinear, and machine learning algorithms, to estimate pig weight \citep{li2014estimation}.


\section{Technological Frameworks}

The manual techniques used in the majority of livestock weight estimation systems today have given way to more automated, technologically advanced systems. Physical weighing scales, which were labor-intensive, time-consuming, and frequently uncomfortable for animals, were the foundation of traditional methods. On the other hand, cutting-edge image technologies are now used in modern systems, including 3D depth sensors like the Microsoft Kinect and binocular and monocular vision systems. For extremely precise 3D reconstructions of animals, binocular vision systems employ two cameras to collect stereo images; however, the expense and complexity of these installations may make them impractical for small farms \citep{rousing2004stepping}. Compared to binocular vision systems, monocular vision systems, which use single-camera setups, are more affordable but have worse precision.

Smart sensors are at the vanguard of revolutionizing precision agriculture by giving farmers access to real-time information on vital parameters including plant health, temperature, humidity, and soil moisture. With their sophisticated detection methods, these sensors assist farmers in making well-informed decisions that increase crop output. Farmers may automate and optimize tasks like nutrient application and irrigation scheduling by combining these sensors with Internet of Things (IoT) devices and artificial intelligence (AI). By giving real-time data on vital elements like soil moisture, temperature, humidity, and plant health, smart sensors are transforming precision agriculture. Farmers can better control fertilizers, optimize irrigation, and monitor crop conditions by integrating these sensors with IoT and AI technologies. This contributes to resource conservation, increased crop yields, and addressing environmental issues \citep{soussi2024smart}. In general, smart sensors assist in addressing issues related to global agriculture, including depletion of resources, climate change, and rising food production demands.

In general, smart sensors help address problems associated with global agriculture, such as resource depletion, climate change, and increased need for food production. Furthermore, these sensors have started to become extremely important in cattle farming. For example, smart sensors assist farmers in ensuring the comfort and welfare of their livestock by keeping an eye on environmental factors like temperature and air quality in animal housing. Thus, healthier animals and lower veterinary expenses can be achieved by preventing heat stress and disease outbreaks\citep{terence2024systematic}.

The physical characteristics of cattle, such as weight and body dimensions, can be monitored non-invasively using Computer Vision (CV)-Based Sensors, which are frequently combined with AI algorithms. Since no direct contact is required, methods like RGB picture analysis and 3D point cloud offer great accuracy and can lessen animal suffering \citep{ma2024computer}. These technologies have the potential to greatly increase the accuracy of data gathered for monitoring, enabling better cattle care and management. However, small-scale farms face difficulties due to the complexity of establishing computer vision systems, which sometimes require pricey hardware and intricate data processing \citep{terence2024systematic}.  High initial investments are needed for computer vision systems, especially those that use 3D cameras and intricate algorithms. Particularly for smaller farms, the expense of high-quality technology (such as RGB cameras or 3D scanners) and the required computer infrastructure can be prohibitive. 

CV-based systems not only provide budgetary difficulties but also demand specialized staff to properly run and maintain the technology. Farmers may find it difficult to handle system calibrations and solve problems if they lack technical skills. This could lead to erroneous data collecting and less-than-ideal herd management results. Furthermore, the accuracy of the system may be reduced in real-world situations due to environmental conditions like dust, lighting, or even the movement of the animals that impair sensor performance \citep{ma2024computer}. Research is being done to make these technologies more resilient and affordable as they develop, so smaller operations can use them. Nonetheless, these technologies have a great deal of promise to improve livestock monitoring by lowering stress and increasing management effectiveness, even in the face of obstacles.

\section{Image Processing and Machine Learning for Weight Estimation}

\subsection{Historical Development}

In the past, weight estimation was primarily performed manually. A common method used was Body Condition Scoring (BCS), which assesses the fatness or thinness of livestock through visual inspection and tactile assessment (Bercovich et al., 2013). However, BCS is prone to errors and biases, as it relies heavily on the evaluator's judgment, making it less reliable.

In recent years, 3D modeling has emerged as a more accurate alternative for weight estimation. Unlike 2D images, 3D modeling captures volumetric data that correlates more precisely with the actual weight of livestock (Liu et al., 2019). The integration of 3D vision cameras with convolutional neural networks (CNNs) has further improved the accuracy and automation of weight estimation systems. CNNs excel at learning and classifying features from images, which enhances the precision of weight estimates in livestock.

\subsection{Modern Techniques}

Body weight (BW) prediction in livestock can be modeled using four main approaches of increasing complexity: Traditional Approach, Computer Vision Approach, Computer Vision with Machine Learning, and Computer Vision with Deep Learning. The Traditional Approach relies on manually collected morphometric measurements such as heart girth, wither height, and body length, which are then applied in traditional regression models. This method has been widely used for species such as cattle, pigs, sheep, goats, camels, and yaks \citep{franco2017evaluation}; \citep{fadlelmoula2020prediction}; \citep{yan2019body}. While effective, it is labor-intensive and causes stress to the animals. To mitigate these issues, the Computer Vision (CV) Approach uses 2D and 3D electro-optical sensors, such as RGB or Kinect cameras, to capture images for morphometric measurements \citep{ozkaya2013prediction}. Although 3D cameras improve precision, they are expensive and require complex data processing. The CV with Machine Learning (CV+ML) Approach enhances the CV method by automating feature selection with machine learning techniques, although some manual processes, such as image segmentation and morphometric extraction, remain necessary \citep{tasdemir2019ann}. Finally, the CV with Deep Learning (CV+DL) Approach uses deep learning models, including convolutional neural networks (CNNs) and recurrent convolutional networks (RCNNs), to fully automate the BW prediction process \citep{gjergji2020deep}. While this approach has shown significant improvements, challenges still exist in precisely segmenting animals from complex backgrounds \citep{shukla2016metric}. These approaches highlight the evolution of BW prediction models, transitioning from manual methods to fully automated systems, with deep learning offering promising advancements.

\subsection{Challenges in Weight Estimation}

Weight estimation is a complex process that involves integrating multiple factors to ensure accuracy and effectiveness. In precision livestock farming, intelligent perception plays a crucial role in achieving reliable results. This includes various perception and management tasks designed to monitor and assess livestock health and performance \citep{bahlo2019role}.

One of the main challenges in livestock weight estimation is the variability in animal posture and movement. Since animals rarely maintain a consistent posture during measurement, this can lead to inaccurate assessments of their body condition. A common method for estimating an animal's fat or thinness is Body Condition Scoring (BCS), which evaluates fat reserves through visual and tactile assessments \citep{bercovich2013development}. While BCS is widely recognized for its effectiveness, it has limitations. The manual scoring process, which relies on the expertise of trained scorers, is inherently subjective and prone to inconsistency \citep{gjergji2020deep}. Human judgment introduces variability and potential bias, making the results less precise. The accuracy of BCS can vary depending on the scorer’s experience, perception, and environmental conditions, highlighting the need for more objective, automated methods for assessing livestock conditions.

Another challenge in weight estimation is the complexity of farm environments. These environments often lack the stable conditions necessary for accurate data collection. For example, animal features can appear differently based on posture and lighting \citep{ruchay2022live}. Inconsistent light sources can interfere with 3D imaging systems, while animal movements can shift their posture, resulting in incomplete or erroneous data.

\subsection{Applications of Kinect in Agriculture}

In livestock farming, Kinect sensors are used to monitor animal behavior, health, and growth through 3D imaging. This technology allows for early detection of health issues by analyzing subtle changes in posture, gait, and movement patterns. It can identify lameness in cattle early on, allowing for prompt intervention \citep{singh2022smart}. Moreover, Kinect’s non-invasive methods provide automated weight estimation by capturing 3D images, reducing the need for stressful manual weighing processes. Continuous monitoring helps track growth and optimize feeding schedules, contributing to the overall welfare of livestock. In addition to health monitoring, Kinect sensors contribute to environmental control within livestock facilities, tracking variables like temperature and humidity to ensure optimal conditions. In poultry farming, Kinect can monitor flock movement patterns, alerting farmers to potential issues like overcrowding, which could lead to health or productivity problems.

The integration of Kinect into Precision Livestock Farming (PLF) systems further enhances its value, as it collects real-time data alongside other technologies like RFID and GPS. This combination allows for more efficient resource management, improved reproductive tracking, and more precise feeding practices \citep{monteiro2021precision}. Kinect technology is also being used in crop management. Its 3D imaging capabilities are beneficial in monitoring plant growth, detecting diseases early, and optimizing resource use like water and fertilizers. In automated harvesting, Kinect sensors guide robots through fields, identifying ripe crops based on their size and shape. This reduces the need for manual labor and ensures timely harvesting to maximize yield and quality \citep{singh2022smart}.

\section{Synthesis Matrix}

The synthesis matrix provides valuable insights into pig weight estimation methods using various imaging and analysis techniques. Together, these studies provide a cohesive foundation for applying Kinect technology, advanced imaging, and regression methods in the current study, demonstrating approaches for accurate, adaptable, and practical pig weight estimation.
\newpage
\begin{longtable}{|p{0.15\linewidth} | p{0.15\linewidth} | p{0.2\linewidth} | p{0.2\linewidth} | p{0.15\linewidth}|}
	\caption{Synthesis Matrix}
	\label{tab:synthesis matrix}\\
	\hline
	\textbf{Title} & \textbf{Author(s), Year} & \textbf{Methodology} & \textbf{Findings} & \textbf{Relevance} \\
	\hline
	On-barn pig weight estimation based on body measurements by a Kinect v1 depth camera
	& 
	Pezzuolo, A., Guarino, M., Sartori, L., González, L. A., and Marinello, F. (2018) 
	&
	- Kinect v1 Depth Camera\newline
	- Estimated pig weight from body measurements\newline
	- Captured 3D images in a barn environment\newline
	- Extracted length, width, and volume from depth data\newline
	- Developed a regression model for weight estimation
	& 
	Both linear and second-degree regression models showed strong correlations with reference weights, with coefficients of determination above 0.95. The non-linear model reduced the standard error by half, and the second-degree regression model had an absolute error of less than 0.5 kg.
	&
	The current study is closely correlated with this study, especially in terms of the usage of the Kinect v1 camera. This study, however, uses 3D images for the calculation of estimated weights.
	\\
	\hline
	Pig Weight Estimation According to RGB Image Analysis
	& 
	Andras Kárpinszky, Gergely Dobsinszki(2023)
	&
	- RGB Cameras (Dahua Models)\newline
	- RGB image captured above pig pens\newline
	- Mask R-CNN for segmentation\newline
	- Kalman filters for tracking\newline
	- Pretty Contour Picker (PCP) for filtering\newline
	- Weight estimation using Multi-Layer Perceptron (MLP)
	& 
	The system achieved more than 97 percent accuracy in predicting pig weights compared to manually recorded weights. Among the models tested, Model V2 was the most consistent, providing high accuracy across varying weight ranges. The RGB image-based method allows faster and stress-free weight measurement, which is valuable for decision-making in pig farming
	&
	Although a Kinect camera wasn’t used in the study, the setup for data acquisition and processing is similar. Top-view photos of pigs were also used in this study, closely relating to how the current study takes them.
	\\
	\hline
	Estimating Pig Weights from Images without Constraint on Posture and Illumination
	& 
	Kyungkoo Jun, Si Jung Kim, Hyun Wook Ji (2018)
	&
	- 2D Image Processing\newline
	- No relaxed posture and illumination constraints\newline
	- Image processing: binarization, morphological ops, contour analysis\newline
	- Features: area size, curvature, deviation\newline
	- Neural network model trained and tested for weight prediction
	& 
	The study achieved an average estimation error of 3.15 kg and a coefficient of determination (R²) of 0.792. Despite this being lower than previous works, the method was able to estimate pig weights without controlling the environment, posture, or lighting, making it applicable in less constrained settings. The model showed that posture-related features contributed significantly to weight prediction accuracy.
	&
	The study utilized a 2D camera, meanwhile the current study will utilize a Kinect v1 camera. Despite the difference, both studies have similar top-down view camera setups. The idea of estimating pig weights without constraints on posture and illumination can also be an inspiration for future similar undertakings with the Kinect v1 camera.
	\\
	\hline
	Carcass Quality Traits of Fattening Pigs Estimated Using 3D Image Technology
	& 
	A. Peña Fernández, T. Norton, E. Vranken, D. Berckmans (2019)
	&
	- Kinect 3D Cameras\newline
	- Capture 3D top-view images pre-slaughter.\newline
	- Extract image features: lengths, areas, and volumes.\newline
	- MATLAB for stepwise linear regression analysis
	& 
	The regression models achieved an adjusted R² ranging from 70-85 percent during training, but performance decreased to 50-60 percent during validation. The best correlations with slaughter traits, such as final weight and yield, were found using the median values of image features from the last week of the fattening period.
	&
	Considering only the sections relevant to pig weight estimation, the study can be used as a reference for the current study given that it also uses Kinect cameras for image capture. The methods for feature extraction and pig weight estimation and analysis can be used as one of the bases for the methods that will be used in the current study.
	\\
	\hline
\end{longtable}

The synthesis matrix provides valuable insights into pig weight estimation methods using various imaging and analysis techniques. The first study by \citep{pezzuolo2018barn} effectively demonstrates using a Kinect v1 depth camera to capture 3D body measurements, achieving high prediction accuracy with regression models, particularly a second-degree regression model that reduced error to under 0.5 kg. This finding supports the current study’s use of Kinect v1 for accurate weight estimation via 3D imaging. 

In another study, \citep{karpinszky2023pig} employed RGB cameras combined with segmentation (Mask R-CNN) and MLP models, achieving over 97 percent accuracy in weight prediction. Although RGB cameras differ from Kinect, the segmentation and tracking methods align well with the current study’s top-view imaging approach and offer potential techniques for data processing in the project.

The work of \citep{jun2018estimating} further broadens applicability by achieving weight estimation without controlling for pig posture or lighting, using a neural network model on 2D images with a 3.15 kg error margin. This flexibility offers useful insights into handling environmental variability, which could enhance the robustness of your Kinect-based approach. 

Lastly, \citep{pena2019carcass} applied Kinect-based 3D imaging for carcass trait prediction, with results showing time-evolving feature accuracy between 50-85 percent. Their regression techniques could be useful for refining the project’s model to predict weight accurately.
Together, these studies provide a cohesive foundation for applying Kinect technology, advanced imaging, and regression methods in the current study, demonstrating approaches for accurate, adaptable, and practical pig weight estimation.

}